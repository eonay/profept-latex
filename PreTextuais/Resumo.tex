%--------------------------------------------------------------------------
%--------------------- Resumo em Português --------------------------------
%--------------------------------------------------------------------------

\setlength{\absparsep}{18pt} % ajusta o espaçamento dos parágrafos do resumo
\begin{resumo}
O ensino mediado por elementos interativos em plataformas digitais tem como objetivo promover o engajamento dos estudantes e auxiliar na prática docente. Nesse contexto, a ferramenta \textbf{Mathix} foi desenvolvida para enfrentar as dificuldades recorrentes no ensino de matrizes, relacionadas à abstração matemática e à ausência de recursos interativos em métodos tradicionais. As tecnologias digitais educacionais têm demonstrado potencial para tornar o aprendizado mais dinâmico, acessível e alinhado às novas demandas tecnológicas, contribuindo para um ensino mais inclusivo. O \textbf{Mathix} oferece uma interface intuitiva e amigável, permitindo o uso de exercícios interativos e recursos visuais que auxiliam os docentes a explorar e manipular matrizes de forma prática e integrada ao aprendizado teórico. A metodologia de desenvolvimento incluiu uma revisão bibliográfica sobre o ensino de matrizes e o uso de tecnologias educacionais, além de entrevistas com professores para identificar as principais dificuldades enfrentadas no ensino desse tema. Após o desenvolvimento, o aplicativo foi submetido a testes com docentes, utilizando questionários para avaliar sua eficácia, interface e experiência do usuário antes e após o uso da ferramenta. O objetivo é consolidar o \textbf{Mathix} como um recurso inovador no ensino de matrizes, promovendo maior engajamento e aprendizado por meio da integração de práticas interativas e tecnológicas.


 \vspace{\onelineskip}
 \noindent
 \textbf{Palavras-chave}: EPT, Ferramentas Educacionais, Matrizes, Construtivismo, Matemática.
\end{resumo}

%--------------------------------------------------------------------------
%--------------------- Resumo em Inglês --------------------------------
%--------------------------------------------------------------------------
\begin{resumo}[Abstract]
 \begin{otherlanguage*}{english}
Teaching mediated by interactive elements on digital platforms aims to promote student engagement and assist teaching practice. In this context, the \textbf{Mathix} tool was developed to address the recurring difficulties in teaching matrices, related to mathematical abstraction and the lack of interactive resources in traditional methods. Digital educational technologies have shown potential to make learning more dynamic, accessible and aligned with new technological demands, contributing to more inclusive teaching. \textbf{Mathix} offers an intuitive and user-friendly interface, allowing the use of interactive exercises and visual resources that help teachers explore and manipulate matrices in a practical way that is integrated with theoretical learning. The development methodology included a literature review on the teaching of matrices and the use of educational technologies, as well as interviews with teachers to identify the main difficulties faced in teaching this subject. After development, the application was tested with teachers, using questionnaires to evaluate its effectiveness, interface and user experience before and after using the tool. The aim is to consolidate \textbf{Mathix} as an innovative resource for teaching matrices, promoting greater engagement and learning through the integration of interactive and technological practices.


   \vspace{\onelineskip}
   \noindent 
   \textbf{Keywords}: EFA, Educational Tools, Matrices, Constructivism, Mathematics.
 \end{otherlanguage*}
\end{resumo}