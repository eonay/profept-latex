%-------------------------------------------------------------
%---------------------- Apêndices ----------------------------
%-------------------------------------------------------------

\begin{apendicesenv}
\partapendices  % Indica o início dos Apendices
\chapter{Termo de Consentimento Livre e Esclarecido (TCLE)}

\begin{center}
\textbf{Termo de Consentimento Livre e Esclarecido (TCLE)}
\end{center}


Prezado (a) Senhor (a):
Gostaríamos de convidar sob sua responsabilidade para participar do Projeto de pesquisa do Instituto Federal do Amapá - IFAP. O Projeto com o título Práticas de Ensino com Matrizes, será realizado no Câmpus Macapá, sendo conduzido pelo pesquisador identificado neste documento. O objetivo e mapear as ferramentas mais aplicadas para ensino de matrizes; analisar o uso de ferramentas aplicadas com matrizes; classificar recursos aplicados a matrizes nos planos de aula.

Os sujeitos pesquisados serão os docentes da unidade de Macapá-AP
do Curso de Licenciatura em Matemática. A participação docente é muito importante e ela se daria no ano de 2023 e ou 2024.
Logo em seguida será solicitado a cada participante que responda um breve questionário.

Esclarecemos que a participação do docente é totalmente voluntária, podendo o(a) senhor(a) solicitar a recusa ou desistência de participação a qualquer
momento, sem que isto acarrete qualquer ônus ou prejuízo. Esclarecemos,
também, que as informações prestadas serão utilizadas somente para os fins de pesquisa (ou para esta e futuras pesquisas) e serão tratadas com o mais absoluto sigilo e confidencialidade, de modo a preservar a identidade do docente. Os dados levantados ficarão sob a guarda do pesquisador por um período de cinco anos e, após esse tempo, os instrumentos de coletas de dados em mídia e ou em papel serão picotados/destruídos e encaminhados à reciclagem. Em relação aos riscos dos participantes da pesquisa podem se sentir inibidos durante a participação ao responder ao questionário. Objetivando evitar ou diminuir tais riscos aos docentes serão avisados que podem solicitar esclarecimento de qualquer dúvida (antes, durante ou depois da pesquisa), de forma individual, ou mesmo desistir da participação a qualquer momento. Esclarecemos ainda, que nem o(a) senhor(a) pagarão ou serão remunerados (as) pela participação, sendo a mesma de forma voluntária.



Garantimos, no entanto, que todas as despesas decorrentes da pesquisa serão financiadas pelo pesquisador. Como benefício gerado por essa pesquisa, espera-se com a conclusão deste projeto, inicialmente incentivar e promover a adoção de produtos educacionais na prática docente assim como métodos científicos e de pesquisa para a produção.



Contudo, tenho sido devidamente esclarecido sobre os procedimentos da
pesquisa, concordo com a participação voluntariada sob minha responsabilidade na pesquisa descrita acima. As suas respostas  não serão divulgadas de forma a possibilitar a identificação, sendo guardadas em sigilo.

Caso o(a) senhor(a) tenha dúvidas ou necessite de maiores esclarecimentos poderá contactar o pesquisador responsável pelo e-mail eonay.web@gmail.com, ou por telefone (96) 99122-4131.



\hfill \break
\begin{center}
\rule{90mm}{1pt}

Participante Voluntário(a) [Assinatura]
\end{center}

\hfill \break
\begin{center}
\rule{90mm}{1pt}

Telefone e ou E-mail do Participante Voluntário
\end{center}


\hfill \break

Em caso de dúvidas com respeito aos aspectos éticos deste estudo, poderei consultar:
\newline
\newline
Comitê de Ética em Pesquisa da Universidade Estadual do Amapá - CEP/UEAP
\newline 
E-mail: cep@ueap.edu.br
\newline 
Endereço: Avenida FAB esquina com Tiradentes, Centro, AP, CEP: 68.900-098
\newline 
Telefone: (96) 99116-9811
\newline
\hfill
\newline
Pesquisador(a): Eonay Barbosa Gurjão
\newline
Telefone para contato: (96) 99122-4131
\newline
E-mail para contato: eonay.web@gmail.com
\newline
Orientador(a): Klenilmar Lopes Dias
\newline
\newline
Número do CAAE\footnote{Certificado de Apresentação de Apreciação Ética}: 70930823.6.0000.0211







\chapter{Ficha de Avaliação de Experiência do Usuário}


\begin{center}
\textbf{
Ficha de Avaliação de Experiência do Usuário\\
Adaptado de \citeonline{laugwitz2008construction}
}
\end{center}
E-mail do Avaliador:\hrulefill \\
Usuário Avaliador: \hrulefill \\

\begin{center}
Produto Avaliado: Mathix    
\end{center}


Considerando os itens que mensuram diretamente a atratividade visual, bem como a qualidade dos aspectos ergonômicos, defina uma opção de acordo com a sua experiência de uso do produto educacional:





\begin{enumerate}
    
    \item Atratividade\\Você gostou ou não do produto?
    \begin{enumerate}
        \item ( ) Desagradável OU ( ) Agradável
        \item ( ) Feio OU ( ) Atraente
        \item ( ) Desinteressante OU ( ) Atrativo
        \item ( ) Bom OU ( ) Ruim
        \item Comentários sobre a Atratividade:\hrulefill \\     
        \rule{14cm}{.1pt}\\
        \rule{14cm}{.1pt}\\
            
    \end{enumerate}

    
    \item Controle\\Você se sente no controle da situação durante a interação?
    \begin{enumerate}
        \item ( ) Não atende às expectativas OU ( ) Atende as expectativas
        \item ( ) Imprevisível OU ( ) Previsível
        \item ( ) Desinteressante OU ( ) Atrativo
        \item ( ) Inseguro OU ( ) Seguro
        \item Comentários sobre o Controle:\hrulefill \\     
        \rule{14cm}{.1pt}\\
        \rule{14cm}{.1pt}\\
              
    \end{enumerate}


    \item Eficiência\\O produto pode ser utilizado de maneira fácil e eficiente?
    \begin{enumerate}
        \item ( ) Ineficiente OU ( ) Eficiente
        \item ( ) Impraticável OU ( ) Prático
        \item ( ) Desorganizado OU ( ) Organizado
        \item( ) Lento OU ( ) Rápido
        \item Comentários sobre o Eficiência:\hrulefill \\     
        \rule{14cm}{.1pt}\\
        \rule{14cm}{.1pt}\\
        
    \end{enumerate}


    \item Estimulação\\Você se sente motivado a utilizar o produto novamente?
    \begin{enumerate}
        \item ( ) Aborrecido OU ( ) Excitante
        \item ( ) Desinteressante OU ( ) Interessante
        \item ( ) Desmotivante OU ( ) Motivante
        \item ( ) Sem valor OU ( ) Valioso
        \item Comentários sobre a Estimulação:\hrulefill \\      
        \rule{14cm}{.1pt}\\
        \rule{14cm}{.1pt}\\
        
    \end{enumerate}
    
    \item Novidade\\O produto é inovador e criativo?
    \begin{enumerate}
        \item ( ) Comum OU ( ) Vanguardista
        \item ( ) Conservador OU ( ) Inovador
        \item ( ) Sem criatividade OU ( ) Criativo
        \item ( ) Convencional OU ( ) Original
        \item Comentários sobre a Novidade:\hrulefill \\     
        \rule{14cm}{.1pt}\\
        \rule{14cm}{.1pt}\\
    \end{enumerate}

    \item Perspicuidade\\O produto é fácil de entender e de se familiarizar?
    \begin{enumerate}
        \item ( ) Complicado OU ( ) Fácil
        \item ( ) Incompreensível OU ( ) Compreensível
        \item ( ) De difícil aprendizagem OU ( ) De fácil aprendizagem
        \item ( ) Confuso OU ( ) Evidente 
        \item Comentários sobre a Perspicuidade:\hrulefill \\       
        \rule{14cm}{.1pt}\\
        \rule{14cm}{.1pt}\\
        
    \end{enumerate}

    
\end{enumerate}

Por fim, atribua uma pontuação a fim de definir o Grau de Experiência de uso do produto avaliado.
PONTUAÇÃO FINAL DA AVALIAÇÃO: NOTA (0 à 10) NOTA:(\space\space\space\space)
\\
\\
\\
   PARECER FINAL DO(A) AVALIADOR(A):\hrulefill \\
        \rule{16cm}{.1pt}\\
        \rule{16cm}{.1pt}\\
        \rule{16cm}{.1pt}\\
        \rule{16cm}{.1pt}\\
        

















\chapter{Ficha de Avaliação de Interface para Usuário}

\begin{center}
\textbf{
Ficha de Avaliação de Interface para Usuário\\
Adaptado de \citeonline{preece2015interaction}
}
\end{center}
E-mail do Avaliador: \hrulefill \\
Usuário Avaliador: \hrulefill \\

\begin{center}
Produto Avaliado: Mathix    
\end{center}


Considerando os itens que mensuram diretamente a satisfação de interação com a interface para uso do produto educacional:




\begin{enumerate}
    
    \item Quão fácil foi instalar o nosso o Mathix?
    \begin{enumerate}
        \item ( ) Extremamente fácil
        \item ( ) Moderadamente fácil
        \item ( ) Nada fácil\\
    \end{enumerate}


    \item Quão fácil de ser usada é a interface do software?
    \begin{enumerate}
        \item ( ) Extremamente fácil
        \item ( ) Muito fácil
        \item ( ) Moderadamente fácil
        \item ( ) Não muito fácil
        \item ( ) Nada fácil            
    \end{enumerate}


    \item Com que frequência o Mathix costuma congelar ou falhar?
    \begin{enumerate}
        \item ( ) Constantemente
        \item ( ) Frequentemente
        \item ( ) Ainda não sei (usuário novo)
        \item ( ) Ocasionalmente
        \item ( ) Nunca          
    \end{enumerate}


    \item Qual é a sua opinião geral sobre o desempenho do Mathix?
    \begin{enumerate}
        \item ( ) Ótimo
        \item ( ) Bom
        \item ( ) Regular
        \item ( ) Ruim
        \item ( ) Péssimo
    \end{enumerate}


    \item Qual é a probabilidade de você recomendar o software para outros docentes?
    \begin{enumerate}
        \item ( ) Extremamente alta 100\%
        \item ( ) Muito alta 75\%
        \item ( ) Talvez sim, talvez não 50\%
        \item ( ) Baixa 25\%
        \item ( ) Nenhuma 0\%
    \end{enumerate}


    \item Por favor, diga-nos em suas próprias palavras, quais aspectos do Mathix precisamos melhorar (em 500 caracteres):\hrulefill \\
    \rule{15cm}{.1pt}\\
    \rule{15cm}{.1pt}\\
    \rule{15cm}{.1pt}\\
    
\end{enumerate}

Por fim, atribua uma pontuações para definir o Grau de Satisfação para com uso do produto avaliado.
PONTUAÇÃO FINAL DA AVALIAÇÃO: NOTA (0 à 10) NOTA:(\space\space\space\space)
\\
\\
\\
   PARECER FINAL DO(A) AVALIADOR(A):\hrulefill \\
        \rule{16cm}{.1pt}\\
        \rule{16cm}{.1pt}\\
        \rule{16cm}{.1pt}\\
        \rule{16cm}{.1pt}\\
        









\chapter{Ficha de Avaliação de Pré Diagnostico e Sócio Tecnológica}

\begin{center}
\textbf{
Ficha de Avaliação de Pré Diagnostico e Sócio Tecnológica
}
\end{center}
E-mail do Avaliador:\hrulefill \\
Usuário Avaliador:\hrulefill \\

\begin{center}
Avaliado Individual    
\end{center}


Considerando os itens que mensuram diretamente sua prática docente responda:




\begin{enumerate}

    \item Sua idade em anos (número)?
    \begin{enumerate}
        \item \rule{4cm}{.1pt}\\
    \end{enumerate}


    \item Seu gênero?
    \begin{enumerate}
        \item ( ) Masculino
        \item ( ) Feminino
    \end{enumerate}
    
    
    \item Já trabalhou com o conteúdo de matrizes em alguma disciplina?
    \begin{enumerate}
        \item ( ) Não
        \item ( ) Sim
    \end{enumerate}


    \item Conhece alguma software/aplicativo para trabalhar com matrizes?
    \begin{enumerate}
        \item ( ) Não
        \item ( ) Sim, quais são? \hrulefill \\
          
    \end{enumerate}


    \item Possui acesso a internet?
    \begin{enumerate}
        \item ( ) Apenas no trabalho
        \item ( ) Apenas em casa
        \item ( ) Ambos (casa e trabalho)
        \item ( ) Sem acesso a internet
    \end{enumerate}


    \item Qual o dispositivo mais utilizado para acesso a internet?
    \begin{enumerate}
        \item ( ) Mobile (smatphone)
        \item ( ) Computador de mesa (desktop)
        \item ( ) Computador portátil (laptop ou notebook)
        \item ( ) Tablet
        \item ( ) Outros
    \end{enumerate}


    \item Já utilizou algum software/aplicativo em sua prática docente?
    \begin{enumerate}
        \item ( ) Nunca
        \item ( ) Raramente
        \item ( ) Às vezes
        \item ( ) Muito
        \item ( ) Sempre
    \end{enumerate}


    \item Na sua perspectiva é possível ensinar o conteúdo básico de matrizes com software aplicativo?
    \begin{enumerate}
        \item ( ) Abstenho-me
        \item ( ) Concordo completamente
        \item ( ) Discordo completamente
    \end{enumerate}


    \item Como você avalia o uso de softwares/aplicativos educacionais no ensino?
    \begin{enumerate}
        \item ( ) Insatisfatório
        \item ( ) Regular
        \item ( ) Satisfatório
        \item ( ) Muito Satisfatório
    \end{enumerate}
    
    
\end{enumerate}






\chapter{Ficha de Avaliação do Produto Educacional}

\begin{center}
\textbf{
Ficha de Avaliação do Produto Educacional
}
\end{center}
E-mail do Avaliador:\hrulefill \\
Usuário Avaliador: \hrulefill \\

\begin{center}
Avaliado Individual  
\end{center}


Considerando os itens que mensuram diretamente o produto educacional Mathix, responda:


\begin{enumerate}

    \item Uma nota de 0 à 10 para produto educacional?
    \begin{enumerate}
        \item \rule{4cm}{.1pt}\\
    \end{enumerate}


    \item O produto educacional deve ser utilizado no ensino?
    \begin{enumerate}
        \item ( ) Sim
        \item ( ) Não
    \end{enumerate}
    
    
    \item Você utilizaria o produto educacional em sua prática docente?
    \begin{enumerate}
        \item ( ) Não
        \item ( ) Sim
    \end{enumerate}


    \item De forma geral, sua satisfação ao utilizar o produto educacional na forma software/aplicativo?
    \begin{enumerate}
        \item ( ) Insatisfatório
        \item ( ) Regular
        \item ( ) Satisfatório
        \item ( ) Muito Satisfatório
    \end{enumerate}


    \item Surgiu alguma dificuldade em manusear o produto educacional?
    \begin{enumerate}
        \item ( ) Sim
        \item ( ) Não
    \end{enumerate}


    \item A aplicação propôs a fazer a função para qual foi desenvolvido?
    \begin{enumerate}
        \item ( ) Insatisfatório
        \item ( ) Regular
        \item ( ) Satisfatório
        \item ( ) Muito Satisfatório
    \end{enumerate}
    

   \item Por favor, diga-nos em suas próprias palavras, quais aspectos no produto educacional precisamos melhorar (em 500 caracteres):\hrulefill \\
    \rule{15cm}{.1pt}\\
    \rule{15cm}{.1pt}\\
    \rule{15cm}{.1pt}\\




    
\end{enumerate}





\chapter{Descrição Técnica do Produto Educacional}

\begin{center}
    \textbf{ Descrição Técnica do Produto Educacional}
\end{center}

\begin{flushleft}


\textbf{Origem do produto}: Trabalho de Dissertação “Prática de Ensino com Matrizes”.

\textbf{Área de conhecimento}: Ensino. Linha de Práticas Educativas em EPT.

\textbf{Público Alvo}: Professores da Educação Básica, Técnica e Tecnológica - EBTT e demais interessados em temas referentes ao trabalho docente com ferramentas educacionais.

\textbf{Categoria deste produto}: Ferramenta educacional aberta para auxilio no ensino introdutório de matrizes.

\textbf{Estruturação do Produto}: Este produto é uma software multiplataforma para interação professor e aluno para prática pedagógica interativa dinamicamente adaptada.

\textbf{Registro do Produto/Ano}: Biblioteca do Campus Santana do IFAP, 2024.

\textbf{Avaliação do Produto}: 3 (três) professores que compuseram a Banca de Defesa da Dissertação.

 \textbf{Disponibilidade}: Irrestrita, preservando-se os direitos autorais bem como a proibição do uso comercial do produto.

\textbf{Divulgação}: Em formato digital.

\textbf{Instituições envolvidas}: Instituto Federal do Amapá, Campus Santana e Campus Macapá.

\textbf{URL}: http://eonay.ddns.net:8000

\textbf{Idioma}: Português

\textbf{Cidade}: Santana

\textbf{País}: Brasil
\end{flushleft}




\end{apendicesenv}