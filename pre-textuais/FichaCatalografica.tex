% ---------------------------------------------------------------
% ----------------  Ficha Catalográfica  -------------------------
% ---------------------------------------------------------------
% Modelo de ficha catalográfica. Você deverá substituir esta
% folha na versão final da monografia por um pdf fornecido pela 
% biblioteca. Salve o modelo oficial como ficha_catalografica.pdf
% e use o comando abaixo para inseri-lo na versão final do texto.

%\begin{fichacatalografica}
%    \includepdf{ficha_catalografica.pdf}
%\end{fichacatalografica}



%% Modelo de Como fazer a Ficha Catalográfica:

\begin{fichacatalografica}
	\sffamily
	\vspace*{\fill}					% Posição vertical
	\begin{center}					% Minipage Centralizado
	\fbox{\begin{minipage}[c][8cm]{13.5cm}		% Largura
	\small
	\imprimirautorcite.
	%Sobrenome, Nome do autor
	
	\hspace{0.5cm}  \\
	\imprimirtitulo  / \imprimirautor. --, \imprimirano-
	
	\hspace{0.5cm} \pageref{LastPage} p. 1 :il. (colors; grafs; tabs).\\
	
	\hspace{0.5cm} \imprimirorientadorRotulo~\imprimirorientador\\
	
	\hspace{0.5cm}
	\parbox[t]{\textwidth}{\imprimirtipotrabalho~-~\imprimirinstituicao. ~\\
	\imprimirinstituto. \\ ~\imprimirdepartamento.}\\
	
	\hspace{0.5cm}
		1. EPT.
		2. Matrizes.
		3. Matemática.
        4. Ferramentas Educacionais.
        5. Construtivismo.			
	\end{minipage}}
	\end{center}
\end{fichacatalografica}


