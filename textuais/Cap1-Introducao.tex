\chapter{Introdução}
\label{introducao}


No cenário atual da educação, a incorporação de tecnologias inovadoras é fundamental para aprimorar as práticas docentes e enriquecer a experiência de aprendizado. Nesse contexto, este trabalho apresenta o desenvolvimento de um software específico para auxiliar no ensino da matemática, com foco no ensino profissional e tecnológico, especialmente na introdução ao conteúdo de matrizes. O produto educacional foi projetado para aumentar a eficácia no aprendizado dos estudantes, simplificando conceitos complexos. Além disso, serve como uma ferramenta de apoio para que os educadores explorem e ampliem suas abordagens pedagógicas.

Por meio de uma interface intuitiva e recursos interativos, o software busca oferecer uma nova perspectiva para o ensino e a compreensão do conteúdo de matrizes, em ambientes acadêmicos formais e informais. Desde o ensino de conceitos fundamentais até a aplicação em contextos de exercícios, cada aspecto do software foi meticulosamente projetado para promover a compreensão profunda conceitual dos princípios matemáticos do uso de matrizes.

A ferramenta oferece uma ampla variedade de exercícios automatizados e dinâmicos, permitindo que os educadores personalizem as atividades de acordo com as necessidades específicas de suas turmas e currículos. Dessa forma, não só se fortalece o aprendizado individualizado, como também se fomenta um ambiente de sala de aula invertida que estimula e proporciona uma descoberta entre alunos e professores.

Com a adoção do software no ensino da matemática, os educadores terão acesso a uma ferramenta poderosa que simplifica a preparação de aulas e potencializa o impacto no desenvolvimento dos alunos, promovendo um aprendizado mais atrativo e interativo.





%Uma abordagem de aprendizagem ativa é uma técnica de ensino baseada em atividades que permite aos alunos participar e, de fato, tornar-se protagonistas no processo de construção de seu próprio conhecimento. Ou seja, são abordagens baseadas menos na transferência de informações e mais no desenvolvimento de habilidades.


%Segundo \citeonline{wenglinsky2005using}, a tecnologia educacional não deve ser observada como um fenômeno isolado, mas deve sim ser considerada uma peça do quebra-cabeça de como os professores ensinam e os alunos aprendem. Assim, muitos pesquisadores educacionais indicam que a integração de tecnologia na sala de aula pode ser vantajosa para alunos e professores. Por exemplo, a tecnologia pode auxiliar a motivar os alunos e proporcionar-lhes habilidades importantes para reforçar seu aprendizado \cite{silva2021integraccao}.



%Dentre as metodologias ativas mais comuns, podemos citar a aprendizagem baseada em problemas, a aprendizagem baseada em equipes, a aprendizagem baseada em projetos, a aprendizagem baseada em letramento informacional, a instrução por colegas e a gamificação. Entretanto, todas elas têm o mesmo fundamento na problematização e visam a aprendizagem significativa \cite{moreira2016accoes}.



%Um diagnóstico inicial focou no plano de curso de Licenciatura em Matemática em face da análise dos documentos públicos do curso, onde algumas características foram levantadas, como os documentos de criação e autorização de curso, informações dos currículos dos docentes, documentos públicos da coordenação e plano pedagógico com os instrumentos ementários assim como os próprios conteúdos.

%No estudo preliminar em análise do plano de curso apenas uma única ferramenta educacional é citada, o MatLab e, com intuito de aprofundar na investigação, partiu-se para procura dos planos de aula, ou seja, o planejamento em detalhes da prática docente onde foi identificado que o mesmo não encontra-se disponível à publico encontra-se é de caráter público assim impossibilitando uma análise completa para afirmar ou aferir o fato, apenas algumas poucas evidências.

%\url{http://www.uel.br/projetos/matessencial/basico/medio/matrizes.html#menu}


%O projeto de pesquisa é na área de educação na subárea de ensino com a temática de ferramentas educacionais aplicadas no currículo do curso e voltado para a componente de matrizes no curso de Licenciatura em Matemática do Instituto Federal do Amapá (IFAP) no Câmpus Macapá na perspectiva docente para adoção nas aulas.


%Busca-se auxiliar na aplicação de Metodologias Ativas com Gameficação no ensino da Matemática. Desenvolver uma aplicação sobre uma plataforma web para utilização na prática do ensino de matrizes e aplicar técnica de Inteligência Artificial com foco em aprendizagem por reforço para uma aprendizagem significativa mediadas por tecnologia digitais da informação e comunicação (TDIC).

%A proposta de pesquisa busca identificar, propor e aplicar metodologias inovativas e tecnológicas na prática docente para uso de ferramentas no ensino, buscando alternativas para as práticas tradicionais na Educação Profissional e Tecnológica (EPT). Nessa concepção de EPT, o ambiente virtual e as novas tecnologias são utilizados para disponibilizar diferentes possibilidades de aprendizagens, para se adaptar às mudanças e atender ao "novo estudante" da era digital, que participa ativamente da construção do conhecimento e frente a isso, o docente deverá rever sua prática pedagógica.


\section{Problemática}

Quais são as características de ferramentas educacionais eficazes para o ensino presencial do conteúdo de matrizes nos cursos de Licenciatura em Matemática?



\section{Hipóteses}
\label{sec:hipoteses}

\begin{itemize}
    \item A aplicação de ferramentas educacionais promove maior engajamento em sala de aula no ensino de matrizes;
    \item Recursos visuais interativos podem melhorar a compreensão do conteúdo e dinamizar a prática docente no ensino de matrizes;
    \item O uso de ferramentas tecnológicas em ambientes virtuais de aprendizagem contribui significativamente para o ensino e a aprendizagem de matrizes;
\end{itemize}



\section{Objetivos}
\label{objetivos}

%O objetivo geral é analisar as ferramentas aplicadas pelos docentes no ensino de matrizes. Os objetivos específico do projeto segue:

O objetivo geral é identificar metodologias inovativas e tecnológicas aplicadas pelos docentes no currículo do curso e voltado para ao componente de matrizes no curso de Licenciatura em Matemática do Instituto Federal do Amapá (IFAP) no Campus Macapá. Os objetivos específico do projeto segue:

\begin{itemize}
    \item Identificar ferramentas educacionais utilizadas no ensino de matrizes na Licenciatura em Matemática do IFAP - Campus Macapá;

    \item Classificar os recursos utilizados para o ensino de matrizes, com foco em metodologias inovadoras e tecnológicas;

    \item Propor uma aplicação baseada em plataforma web com gamificação para o ensino de matrizes;
    
    \item Avaliar o impacto de ferramentas tecnológicas aplicadas ao ensino de matrizes;
    
    \item Propor um modelo de relatório acadêmico em formato LaTeX para o ProfEPT, simplificando a aplicação das normas ABNT, abstraindo suas complexidades.
\end{itemize}




            
\section{Justificativa}
\label{sec:justificativa}

Conteúdo de matrizes é aplicado em diversas disciplinas e cursos, abordado sob diferentes perspectiva e de diferentes aspectos incluindo a aplicação como estrutura de dados, assim tem o potencial de ser trabalhado de forma transdisciplinar. A matriz se aplica em diversas áreas de conhecimento sendo de ciências exatas ou não, onde pode-se aplicar a aprendizagem significativa com outras técnicas em sala e metodologias ativas e criativas dentre outras.

\citeonline{levy1999cibercultura} argumenta que as novas tecnologias devem ser empregadas para enriquecer o ambiente educacional. Para dar conta dessa inserção no cenário educacional é solicitado aos professores novos saberes e competências para lidar criticamente com as Tecnologias de Informação e Comunicação (TIC) em seu dia a dia docente.

Na mesma perspectiva \citeonline{kenski2001direccao} assegura ser necessário ao docente conhecer o computador, os suportes midiáticos e todas as possibilidades educacionais e interativas para aproveitá-las nos suportes midiáticos nas mais variadas situações de ensino-aprendizagem e nas mais diferentes realidades educacionais. O professor passa a ser o encarregado de uma grande responsabilidade de utilizar as TIC como recurso para construir e difundir conhecimentos em sua prática docente.

O uso de ferramentas tecnológicas na educação profissional potencializa o ensino, especialmente para alunos já familiarizados com essas tecnologias, papel docente é mediar o conhecimento aplicando diversos recursos, assim expandido as ações de reflexão e autonomia dos alunos. A prática docente deve adaptar a diferentes possibilidades, formas e formatos e, neste contexto a proposta é voltada a prática inovativa docente adotando elementos para intermediar o ensino.

A proposta tem potencialidade de trabalhar com aprendizagem combinada a sala de aula invertida (\textit{Blended Learning e Flipped Classroom}) podendo ser aplicado para dentro e fora da sala de aula, ou seja, nos espaços tradicionais formais e não formais. Assim podendo viabilizar também de forma a dar autonomia aos estudantes, colocando-os no centro do processo como protagonista das ações, estimulando a pensar e resolver problemas de forma independente.

As metodologias de \textit{blended learning} (no formato híbrido) e \textit{flipped classroom} (ensino invertido) formam perfis de profissionais mais engajados e menos dependentes ou autodidatas para certas atividades, resultando em uma aprendizagem personalizada e efetiva. Restando ao professor a tarefa trivial de mediar o uso e o planejamento dos objetivos de aprendizagem.




















\section{Trabalhos Relacionados}
\label{trabalhos_relacionados}

No trabalho de \citeonline{florencio2021perspectivas} é proposto o uso de uma ferramenta digital (Jamboard) para mediação em aulas remotas na perspectiva docente de caráter qualitativo, aplicado em um estudo de caso. A experiência trouxe mecanismos para vivenciar o ensino remoto utilizando ferramentas tecnológicas por meio de metodologias ativas, ressaltando comparações entre o ensino presencial e o remoto \cite{florencio2021perspectivas}.

Diferentemente do trabalho supracitado a proposta é investigar a prática docente com o uso de ferramentas tecnológicas no ensino presencial e híbrido, com intervenção para avaliar a proposta em conjunto com a componente de matrizes. Assim para uso de novos recursos que favorecem   a   ampliação   das   técnicas   didáticas   pelos   educadores.

Corroborando com a proposta de pesquisa no trabalho de \citeonline{ferramenta_padlet} foca na ferramenta digital Padlet como recurso visual e alternativo para organização de conteúdos para as apresentações de seminários avaliativos como metodologia ativa que promove autonomia e a interação.

Correlacionado com trabalho supracitado, o o uso de matrizes como recurso visual em plataformas web, promovendo autonomia e integração com o conteúdo e de apoio ao docente como recurso educacional aberto. As características do Produto Educacional (PE) abstraem características dos projetos retratados nesta relatório.









\section{Organização do Texto}
\label{organizacao_texto}

Este trabalho está organizado da seguinte forma: inicialmente no Capitulo \ref{introducao} uma breve introdução, seguida do referencial teórico, que apresenta a fundamentação histórica da educação profissional e tecnológica, as tecnologias digitais da informação e comunicação aplicadas a prática docente e a BNCC no Capítulo \ref{fundamentacao}. No Capítulo \ref{metodologia} com percurso metodológico e métodos utilizados. Os resultados e discussões serão tratados no  Capítulo \ref{resultados} e, por fim as considerações finais e limitações no Capítulo \ref{conclusao}.

