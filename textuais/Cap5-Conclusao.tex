
\chapter{Conclusão}
\label{conclusao}

Com os resultados obtidos nesta pesquisa, pode-se concluir que a utilização de software como ferramenta de suporte e apoio no ensino de matrizes é uma estratégia educacional promissora com potencialidade promover um melhor processo de ensino-aprendizagem mais significativo.

O software Mathix permitiu visualizar e manipular de maneira interativa e dinâmica mesmo que em sua versão inicial, o que contribuiu para o desenvolvimento do raciocínio lógico de forma lúdica, criativa e da autonomia dos estudantes durante as atividades. Além disso, pode proporcionar aos alunos uma melhor compreensão do conteúdo de matriz no cotidiano e em sua futura atuação profissional inclusive como futuros docentes.


Assim sendo, é imprescindível reconsiderar as práticas pedagógicas no ensino de matrizes. A utilização da tecnologia deve ser encarada como uma ferramenta benéfica, não como um obstáculo, visando enriquecer as aulas de forma significativa. A adoção de softwares, como o Mathix, pode se transformar em um aliado nesse processo. Dessa forma, o professor pode explorar novos recursos e métodos de ensino, promovendo a eficácia das aulas e captando a atenção dos alunos, resultando em desempenhos positivos.


Em síntese, para que o uso dessas ferramentas educacionais seja eficaz, é crucial que os professores recebam formação contínua, adquirindo o conhecimento necessário para manipular novas tecnologias em suas práticas. Além disso, é essencial que as atividades sejam cuidadosamente planejadas para atender aos diferentes níveis de aprendizagem dos alunos. Por fim, as situações-problema devem ser contextualizadas, relacionando-se com espaços e situações reais, a fim de serem significativas e motivadoras.

O uso de ferramentas de mediação educacionais representam um passo importante na democratização dos espaços e no acesso à tecnologia, embora ainda exija discussões contínuas sobre suas potencialidades e limitações. Compreender as experiências e os conceitos por trás dessas ferramentas para diferentes grupos em cursos é fundamental para direcionar futuras pesquisas.


Portanto, espera-se que o conteúdo desta pesquisa, aliado ao produto educacional desenvolvido, contribua para incentivar a adoção de uma abordagem integrada e abrangente para aplicação de tecnologias na educação, facilitando a aplicação dos conhecimentos científicos adquiridos em outras disciplinas e no mundo de trabalho.



%%========== Section ==========%
\section{Publicações Submetidas}
\label{publicacoes}

A proposta e a metodologia de pesquisa, bem como os resultados obtidos, foram submetidos para avaliação, com o objetivo de fortalecer a credibilidade do estudo. As submissões listadas a seguir ocorreram em momentos distintos, respeitando a ordem cronológica do desenvolvimento da pesquisa.

Os trabalhos resultaram tanto das metodologias propostas quanto dos resultados alcançados, sendo ambos aceitos para apresentação em um evento nacional:

\begin{enumerate}
   \item CNMAC 2023 - XLII Congresso Nacional de Matemática Aplicada e Computacional. 18 à 22 de Setembro de 2023 - Bonito - MS;

   \item SENACEM 2024 - VIII Seminário Nacional do Ensino Médio e III Simpósio Interdisciplinar. 27, 28 e 29 de Novembro de 2024 - Mossoró - RN.
\end{enumerate}




%%========== Section ==========%
\section{Limitações}
\label{limitacoes}

A versão inicial da ferramenta foi projetada para abordar aspectos fundamentais do ensino de matrizes, oferecendo uma base sólida para a futura implementação de novas operações relacionadas a esse tópico. A estrutura desenvolvida busca atender às necessidades iniciais, enquanto mantém espaço para expansão e aprimoramento conforme o uso e \textit{feedback} dos usuários.

Como a aplicação foi desenvolvida no formato de plataforma, a documentação explicativa anexada à ferramenta ainda é limitada. Esse é um ponto a ser aprimorado para garantir que os usuários compreendam plenamente as funcionalidades disponíveis e como utilizá-las de maneira eficiente.

Outro aspecto importante a ser considerado é a capacidade da ferramenta de lidar com múltiplos acessos simultâneos. É essencial realizar testes de desempenho e carga para avaliar a robustez da aplicação em cenários de alta demanda, garantindo sua estabilidade e eficiência mesmo em situações de uso intenso.

%%========== Section ==========%
\section{Trabalhos Futuros}
\label{trabalhos_futuros}

Entre as perspectivas futuras para esta pesquisa, destaca-se a aplicação da matemática em contextos de outros cursos técnicos, visando otimizar estruturas e maximizar o aproveitamento de materiais em diversas áreas. As competências envolvidas possibilitam a exploração desses aspectos de maneira viável, colaborativa e transdisciplinar.

A possibilidade de interação com outras ferramentas educacionais e um incentivo ao uso de tecnologias baseadas em software livre, segundo \citeonline{martins2021software} “software livre” devemos entender aquele software que respeita a liberdade e senso de comunidade dos usuários. A grosso modo, isso significa que os usuários possuem a liberdade de executar, copiar, distribuir, estudar, mudar e melhorar o software.

